\documentclass{article}
\usepackage{CJKutf8}
\begin{document}
\tableofcontents

\section{Utilitaire}
    \subsection{Date et heure}
\begin{CJK}{UTF8}{min}
    \subsubsection{Date}
    \underline{Jour de la semaine (ようび ou 曜日)}~:\\
    \begin{itemize}
        \item lundi 月曜日
        \item mardi 火曜日
        \item mercredi 水曜日
        \item jeudi 土曜日
        \item vendredi 金曜日
        \item samedi 木曜日
        \item dimanche 日曜日
    \end{itemize}
    \underline{les mois~:}\\
    Chiffre suivi de 月 (がつ). \\
    Danger 4 se prononce し et 10 じゅ.
    %TODO Faire un tableau ?
    \underline{Jour:}\\
%    \begin{itemize}
\begin{center}
  \begin{tabular}{ | c | c | c | c |}
    \hline
         1:    &ついたち &
         2:    &ふつか\\ 
         3:    &みか  &
         4:    &よっか\\
         5:    &いつか&
         6:    &むいか\\
         7:    &なのか&
         8:    &ようか\\
         9:    &ここのか &
         10:   &とおか\\
         11:   &十一日 &
         14:   &十よっか \\
         19:   &十く日 &
         20:   &はつか\\
         24:   &二十よっか& 
         29:   &二十く日\\

    \hline
  \end{tabular}
\end{center}
\end{CJK}
    \subsection{positionner des objets/gens}


\section{Particules}


\section{Verbes}
\begin{CJK}{UTF8}{min}
    Les verbes peuvent être séparer en trois groupes.\\
    On peut les distinguers à leur forme neutre.\\
    Les verbes du premier groupe possèdent les terminaisons suivantes:
    う, く, ぐ, す, つ, む, ぬ, ふ, る.\\
    Les verbes du deuxième groupe ont la terminaison suivante:
    る. Ils sont \textbf{toujours} précédés d'une syllabe en i ou e. Cependant certains verbe du groupe I aussi !\\
    Enfin les verbes du derniers groupes sont composés de~:\\
    する (faire) et くる (venir).\\
\begin{center}
\begin{tabular} {|c|c|c|c|}
    \hline
    Forme neutre & Forme Polie & Forme en て & Potentiel \\
    \hline
    G I& &&\\
    BV + う      & BV +$[i]$ +ます &BV +っ +て    &BV +$[e]$+ る\\ 
    BV + つ      & BV +$[i]$ +ます &BV +っ +て    &BV +$[e]$+ る\\ 
    BV + る      & BV +$[i]$ +ます &BV +っ +て    &BV +$[e]$+ る\\
    
    BV + く/ぐ   & BV +$[i]$ +ます &BV +い +て/で &BV +$[e]$+ る\\ 
    
    BV + す      & BV +$[i]$ +ます &BV +し +て    &BV +$[e]$+ る\\ 

    BV + む      & BV +$[i]$ +ます &BV +ん +て    &BV +$[e]$+ る\\ 
    BV + ぬ      & BV +$[i]$ +ます &BV +ん +て    &BV +$[e]$+ る\\ 
    BV + ふ      & BV +$[i]$ +ます &BV +ん +て    &BV +$[e]$+ る\\ 
    \hline
    G II&&&\\
    BV + る & BV + ます & BV+て & BV + られる\\
    \hline
    G III&&&\\
    する &します & して & できる\\
    くる & きます & きて  & こられる\\
    \hline
\end{tabular}
\end{center}
BV: forme neutre sans terminaison\\
$[x]$ veut dire remplacer la dernière syllable de la BV par celle en x.\\

\par
\underline{Exemple~:} \\
\textit{Imperatif:} みてください。\\
\textit{Action en cours:} りんごおをたべています。

\end{CJK}
\section{Adjectif}
\begin{CJK}{UTF8}{min}
Deux cas possibles, les adjectifs en な ou en い.
Utilisation~:\\
Adj + な + nom. \\
Adj + nom. \\

Combinaison~:\\
na + na~: Adj + で + Adj + Nom
i + i~: Adjく + て + Adj + Nom
i + na~: Adjく + て + Adj + な + Nom
na + i~: Adj	+ て + Adj + Nom

%TODO Passé et négation

\end{CJK}
\newpage
\section{Vocabulaire}
%TODO Trier par thème
Notation: le type d'adjectif est noté entre (). De même pour les groupes des verbes.\\
Mot FR (adj ou groupe): Japonais \textit{hiragana}

\begin{CJK}{UTF8}{min}
   \subsection{Salutation et présentation~:}
   \begin{itemize}
       \item  
   \end{itemize}
    \subsection{Indicateur Temporel~:}
    \begin{itemize}
       \item demain: あした
        \item aujourd'hui: きょう
        \item hier: きのう
        \item matin: あさ
        \item soir: よる
        \item ce soir: こんばん
        \item ce matin: けさ
        \item maintenant: いま
        \item la semaine prochainne: らいしゅ
        \item la semaine dernière: せんしゅう
        \item automne: あき
        \item été: なつ
   \end{itemize}
   \subsection{Nourriture~:}
   \begin{itemize}
       \item poire: なち
        \item pêche: もも
        \item pomme: りんご
        \item curry: カレー
        \item gâteau: ケーキ ou おかし (pas tout à fait le même sens)
        \item hamburger: ハンバーガー    
        \item whisky: ウイスキー
        \item café: コーヒー
        \item eau: 水 \textit{みず}
        \item légumes: やさい
        \item fruits: くだもの
        \item thé: おちゃ
        \item viande: にく
        \item viande de cheval: ばにく, viande de boeuf: うしにく, viande de cochon: ふたにく, viande de poulet: とりにく
        \item bonbon: あめ (ton monte)
        \item pastèque: すいか
        \item melon: メロン
        \item cerise: さくらんほ
        \item banane: バナナ
        \item lait: ミルク ou ぎゅうにゅう
        \item petit déjeuner: あさごはん
        \item déjeuner: ひるごはん
        \item souper: ばんごはん
        \item grain de riz: おこめ
        \item fromage: チーズ
   \end{itemize}
   \subsection{Famille~:}
   \begin{itemize}
       \item maman: おかさん (poli) はは (quand on parle de sa mère) 
       \item papa: おとさん (poli) ちち
        \item petit frère: おとうと
        \item grand frère: あに
        \item petite soeur: いもと
        \item grande soeur: あね
   \end{itemize}

   \subsection{Lieu, Transport, Voyage~:}
   \begin{itemize}
       \item ville: まち
        \item bureau de poste: ゆうびんきょく
        \item salon de thé: きっさてん
        \item voiture: くるま
        \item train: でんしゃ
        \item avion: ひこうき
        \item sous marin: せんすいかん
        \item montgolfière: ねつききゅう
        \item métro: ちかてつ ou メトロ
        \item fusée: おうびた
        \item montagne: 山 \textit{やま}
        \item mer: うみ
        \item rivière: かわ
        \item vélo: じてんしゃ
        \item taxi: タクシー
        \item boutique: や (exemple boutique de chuassure: くつや)
        \item boulangerie: パンや, boucherie: にくや, librairie: 本や, fleuriste: はなや
        \item l'arrêt (bus ou autre): のりば
        \item chambre: へや
        \item voyage: りょこう
        \item la conduite: うんてん
        \item hôtel: ホテル
        \item moyen de transport: のりもの
        \item endroit: ところ
   \end{itemize}

   \subsection{Vêtement:}
   \begin{itemize}
       \item couvre chef: ぼうし
       \item chaussette: くつした
        \item chaussure: くつ
        \item cravate: ネクタイ
        \item pull-over: セーター
        \item montre: とけい
        \item lunettes: めがね
        \item chemise: シャツ
        \item montre: とけい
        \item écharpe: マフラー
        \item manteau: こーと
        \item vêtement: ふく
   \end{itemize}

   \subsection{Le corps Humain}
   \begin{itemize}
       \item jambe: あち
   \end{itemize}

    \subsection{Ecole}
   \begin{itemize}
       \item maître: 先生 \textit{せんせい}
       \item dictionnaire: じしょ
       \item encre: インク
       \item lycée: こうこう
       \item examen: しけん
       \item université: だいがく
       \item cité U: だいがくのりょう
       \item cahier: ノート
   \end{itemize}

   \subsection{L'entreprise}
   \begin{itemize}
       \item entreprise: かいしゃ
        \item lieu de travail: しごとば
        \item jours de congé: やすみのひ
        \item client: きゃく (note si on rajoute le préfixe お on obtient l'invité)
   \end{itemize}

   \subsection{Loisir, sport, Hobby~?}
        \begin{itemize}
            \item tableau: え
            \item sport: スポーツ
            \item internet: インターネット
            \item musique: おんがく
            \item jouet: おもちゃ
            \item lecture: どくしょ
            \item natation: 水えい \textit{すいえい}
            \item SF: エスエフ
            \item comédie: コメヂぃー
            \item action: アクション
            \item microphone: マイク
            \item radio: ラジオ
            \item télévision: テレビ
            \item revue: ざっし
            \item film: えいが (e suivit de i => e long /!\\ )
        \end{itemize}
    
    \subsection{Animaux}
    \begin{itemize}
        \item insecte: むし
        \item poisson: さかな
        \item chien: いぬ
        \item chat: 猫 \textit{ねこ}
        \item oiseau: とり
        \item cheval: うま
        \item boeauf: うし
        \item souris/rat: ねずみ
    \end{itemize}

    \subsection{Verbe}
    \begin{itemize}
        \item mourir: しむ
        \item lire: よむ
        \item acheter: 買う \textit{かう} 
        \item écrire: かく
        \item enlever: ぬぐ
        \item manger (II): 食べる (たべる)
        \item voir (II): みる
        \item se lever (II): おきる
        \item se coucher (II): ねる
        \item faire (III): する
        \item venir (III): 来る (くる)
        \item boire: 飲む (のむ)
        \item aller: 行く (いく)
        \item travailler: はたらく
        \item écouter: きく
        \item étudier: べんきょうする
        \item rentrer (I):  かえる
        \item changer (II): かえる
        \item téléphoner: でんわする
        \item être exister: いる (vivant II) ある (inanimé I)
        \item arréter: のる
        \item nager: およぐ
        \item chanter: うたう
        \item conduire: うんてんする
        \item parler: はなす
        \item s'amuser: あそぶ
        \item prendre: とる
        \item s'asseoir: すわる
        \item préparer: つくる
        \item expliquer qqch à qqn: せつめいする
        \item partir: でかける
        \item jouer d'un instrument à vent: ふく à cordes: ひく
        \item se détendre: ゆくりしする
    \end{itemize}
    \subsection{Adjectif}
    \begin{itemize}
        \item froid: さむい
        \item blanc (い): しろい
        \item noir (い): くろい
        \item rouge (い): あかい
        \item bleu (い): あおい
        \item jaune (い): きろい
        \item petit (い): ちいさい
        \item pratique (な): べんり
        \item incommode (な): ふべん
        \item solide (な): しょうぶ
        \item gentil (な): しんせつ
        \item spatieux: ひるかつた
        \item amusant (な): たのし
        \item sérieux (な): まじめ
        \item facile,simple (な): かんたん
        \item élaboré, complexe (な): ふくざん
        \item agréable (な): かいてき
        \item large (い): ひろい
        \item étroit (い): せまい
        \item occupé (い): いそがしい
        \item chaud (い): あつい
        \item mauvais (い): まずい
        \item bon au goût (い): おいちい
        \item intéressant (い): おもしろい
        \item grand (い): 大きい (おおきい)
        \item petit (い): 小さい (ちいさい)
        \item calme (な): しずか
        \item animé (な): にぎやか
        \item ancien (い): ふろい
        \item neuf (い): あたらし
        \item propre (な): きれい
        \item célèbre (い): ゆめい
        \item disponible (な): ひま
        \item chercher: さがす
        \item attendre: まつ
        \item choisir: えらぶ
    \end{itemize}
    \subsection{Autre}
    \begin{itemize}
        \item je ne sais pas: わかりません
        \item chose: もの
        \item journal: しんぶん
        \item lettre: てがみ
        \item intieux (な): ひろかった
        \item pluie: あめ (ton descend)
        \item toujours: いつも
        \item vrai: マル faux: バン
        \item photo: しゃしん
        \item les infos: ニュース
        \item courriel: メル
        \item fleur: はな
        \item tabac: タバコ
        \item neige: ゆき
        \item courage: ゆうき
        \item rêve: ゆめ
        \item anniversaire: おたんじょび
        \item bonne année: しんねん おめでとう ございます (note même structure pour bonne anniversaire)

    \end{itemize}
\end{CJK}
\end{document}
