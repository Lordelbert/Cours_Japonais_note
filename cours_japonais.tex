\documentclass{article}
\usepackage{CJKutf8}
\begin{document}
\tableofcontents

\section{Utilitaire}
    \subsection{Date et heure}
\begin{CJK}{UTF8}{min}
    \subsubsection{Date}
    \underline{Jour de la semaine (ようび ou 曜日)}~:\\
    \begin{itemize}
        \item lundi 月曜日
        \item mardi 火曜日
        \item mercredi 水曜日
        \item jeudi 土曜日
        \item vendredi 金曜日
        \item samedi 木曜日
        \item dimanche 日曜日
    \end{itemize}
    \underline{les mois~:}\\
    Chiffre suivi de 月 (がつ). \\
    Danger 4 se prononce し et 10 じゅ.
    %TODO Faire un tableau ?
    \underline{Jour:}\\
    \begin{itemize}
        \item 1:ついたち
        \item 2:ふつか 
        \item 3:みか
        \item 4:よっか
        \item 5:いつか
        \item 6:むいか
        \item 7:なのか
        \item 8:ようか
        \item 9:ここのか
        \item 10:とおか
        \item 11:十一日
        \item 14: 十よっか 
        \item 19: 十く日
        \item 20: はつか
        \item 24: 二十よっか
        \item 29: 二十く日
    \end{itemize}
\end{CJK}
    \subsection{positionner des objets/gens}


\section{Particules}


\section{Verbes}
\begin{CJK}{UTF8}{min}
    Les verbes peuvent être séparer en trois groupes.\\
    On peut les distinguers à leur forme neutre.\\
    Les verbes du premier groupe possèdent les terminaisons suivantes:
    う, く, ぐ, す, つ, む, ぬ, ふ, る.\\
    Les verbes du deuxième groupe ont la terminaison suivante:
    る. Ils sont \textbf{souvent} précédés d'une syllabe en i ou e.\\
   %  TODO check Danger かえる qui selon si il est considéré du premier ou du second groupe ne possède pas le même sens.
    Enfin les verbes du derniers groupes sont composés de~:\\
    する (faire) et くる (venir).
    
    \subsection{Présent forme polie}
    La forme polie se construit différement selon les groupes.\\
    \par
    \underline{Groupe I~:}
        La terminaison de la forme neutre devient la syllabe correspondante en i puis on ajoute la terminaison ます.\\
        Ainsi る deviens り, む deviens み \ldots\\
        Exemple: chercher: さがす $\rightarrow$ さがします.\\
    \par
    \underline{Groupe II~:}
        La terminaison de la forme neutre `tombe' et on ajoute ます.
        Exemple: manger: たべる $\rightarrow$ たべます.
    \par

    \underline{Groupe III~:} \\
        する $\rightarrow$ します\\
        くる $\rightarrow$ きます
    %TODO Forme negative
    \subsection{Passé}
        Remplacer ます par ました.
    \subsection{forme en té}
    La forme en て permet d'exprimer:
    \begin{enumerate}
        \item la continuité, i.e. être en train de faire quelque chose.
        \item l'impératif poli si suivi de ください.
        \item la profession que l'on exerce
    \end{enumerate}
    Elle permet aussi d'utiliser plus d'un verbe par phrase.
    Elle se construit comme suis:
    \par

    \underline{Groupe I~:}\\
    La terminaison   く $\rightarrow$ いて\\
    La terminaison   ぐ $\rightarrow$ いで\\
    La terminaison   す $\rightarrow$ して\\
    Les terminaisons ふ、む $\rightarrow$ んで\\ 
    Les terminaisons  つ、る、う $\rightarrow$ って
    
\par
    \underline{Groupe II~:}
On enlève le る et on met て à la place.
    \par

    \underline{Groupe III~:}
する $\rightarrow$ して\\
くる $\rightarrow$ きて

    \subsection{Forme negative}
 \underline{Groupe I~:}
    La syllabe devant la terminaison de la forme neutre devient la syllabe correspondante en a.
Sauf pour う qui devient わ.\\
Ensuite on ajoute la terminaison ない.
Exemple: écrire かく $\rightarrow$ かかない. \\
	 acheter かう $\rightarrow$ かわない.
 
    \underline{Groupe II~:}
On enlève le る et on met ない à la place.

    \underline{Groupe III~:}\\
する $\rightarrow$ しない\\
くる $\rightarrow$ こない

\par
\underline{Exemple~:} \\
\textit{Imperatif:} みてください。\\
\textit{Action en cours:} りんごおをたべています。

\end{CJK}
\section{Adjectif}
\begin{CJK}{UTF8}{min}
Deux cas possibles, les adjectifs en な ou en い.
Utilisation~:\\
Adj + な + nom. \\
Adj + nom. \\

Combinaison~:\\
na + na~: Adj + で + Adj + Nom
i + i~: Adjく + て + Adj + Nom
i + na~: Adjく + て + Adj + な + Nom
na + i~: Adj	+ て + Adj + Nom

%TODO Passé et négation

\end{CJK}
\newpage
\section{Vocabulaire}
%TODO Trier par thème
Notation: le type d'adjectif est noté entre (). De même pour les groupes des verbes.\\
Mot FR (adj ou groupe): Japonais \textit{hiragana}

\begin{CJK}{UTF8}{min}
   \subsection{Salutation et présentation~:}
   \begin{itemize}
       \item  
   \end{itemize}
    \subsection{Indicateur Temporel~:}
    \begin{itemize}
       \item demain: あした
        \item aujourd'hui: きょう
        \item hier: きのう
        \item matin: あさ
        \item soir: よる
        \item ce soir: こんばん
        \item ce matin: けさ
        \item maintenant: いま
   \end{itemize}
   \subsection{Nourriture~:}
   \begin{itemize}
        \item bon au goût (い): おいちい
        \item intéressant (い): おもしろい
       \item poire: なち
        \item pêche: もも
        \item curry: カレー
        \item gâteau: ケーキ ou おかし (pas tout à fait le même sens)
        \item hamburger: ハンバーガー    
        \item whisky: ウイスキー
        \item café: コーヒー
        \item eau: 水 \textit{みず}
        \item légumes: やさい
        \item fruits: くだもの
        \item thé: おちゃ
        \item viande: にく
        \item bonbon: あめ (ton monte)
   \end{itemize}
   \subsection{Famille~:}
   \begin{itemize}
       \item mamman: おかさん (poli) はは (quand on parle de sa mère) 
       \item papa: おとさん (poli) ちち
   \end{itemize}

   \subsection{Lieu, Transport, Voyage~:}
   \begin{itemize}
       \item ville: まち
        \item bureau de poste: ゆうびんきょく
        \item salon de thé: きっさてん
        \item voiture: くるま
        \item train: でんしゃ
        \item avion: ひこうき
        \item sous marin: せんすいかん
        \item montgolfière: ねつききゅう
        \item métro: ちかてつ ou メトロ
        \item fusée: おうびた
        \item montagne: 山 \textit{やま}
        \item mer: うみ
        \item rivière: かわ
        \item vélo: じてんしゃ
        \item taxi: タクシー
        \item boutique: や (exemple boutique de chuassure: くつや)
        \item boulangerie: パンや, boucherie: にくや, librairie: 本や, fleuriste: はなや
        \item l'arrêt (bus ou autre): のりば
        \item chambre: へや
        \item voyage: りょこう
        \item la conduite: うんてん
   \end{itemize}

   \subsection{Vêtement:}
   \begin{itemize}
       \item couvre chef: ぼうし
       \item chaussette: くつした
        \item chaussure: くつ
        \item cravate: ネクタイ
        \item pull-over: セーター
        \item montre: とけい
        \item lunettes: めがね
   \end{itemize}

   \subsection{Le corps Humain}
   \begin{itemize}
       \item jambe: あち
   \end{itemize}

    \subsection{Ecole}
   \begin{itemize}
       \item maître: 先生 \textit{せんせい}
       \item dictionnaire: じしょ
       \item encre: インク
       \item lycée: こうこう
        \item examen: しけん
   \end{itemize}

   \subsection{L'entreprise}
   \begin{itemize}
       \item entreprise: かいしゃ
        \item lieu de travail: しごとば
        \item jours de congé: やすみのひ
   \end{itemize}

   \subsection{Loisir, sport, Hobby~?}
        \begin{itemize}
            \item tableau: え
            \item sport: スポーツ
            \item internet: インターネット
            \item musique: おんがく
            \item jouet: おもちゃ
            \item lecture: どくしょ
            \item natation: 水えい \textit{すいえい}
            \item SF: エスエフ
            \item comédie: コメヂぃー
            \item action: アクション
        \end{itemize}
    
    \subsection{Animaux}
    \begin{itemize}
        \item insecte: むし
        \item poisson: さかな
        \item chien: いぬ
        \item chat: 猫 \textit{ねこ}
        \item oiseau: とり
        \item cheval: うま
        \item boeauf: うし
        \item souris/rat: ねずみ
    \end{itemize}

    \subsection{verbe}
    \begin{itemize}
        \item mourir: しむ
        \item lire: よむ
        \item vendre: 買う \textit{かう}
        \item manger (II): 食べる (たべる)
        \item voir (II): みる
        \item se lever (II): おきる
        \item se coucher (II): ねる
        \item faire (III): する
        \item venir (III): 来る (くる)
        \item boire: 飲む (のむ)
        \item aller: 行く (いく)
        \item travailler: はたらく
        \item écouter: きく
        \item étudier: べんきょうする
        \item rentrer (II?):  かえる
        \item téléphoner: でんわする
        \item être exister: いる (vivant II) ある (inanimé I)
        \item arréter: のる
        \item nager: およぐ
        \item chanter: うたう
        \item conduire: うんてんする
        \item parler: はなす
        \item s'amuser: あそぶ
        \item prendre: とる
        \item s'asseoir: すわる
        \item préparer: つくる
        \item expliquer qqch à qqn: せつめいする
        \item partir: でかける
    \end{itemize}
    \subsection{adjectif}
    \begin{itemize}
        \item froid: さむい
        \item blanc (い): しろい
        \item noir (い): くろい
        \item rouge (い): あかい
        \item bleu (い): あおい
        \item petit (い): ちいさい
        \item incommode (な): ふべん
        \item solide (な): しょうぶ
        \item gentil (な): しんせつ
        \item spatieux: ひるかつた
        \item amusant (な): たのし
        \item sérieux (な): まじめ
        \item facile,simple (な): かんたん
        \item élaboré, complexe (な): ふくざん
    \end{itemize}
    \subsection{Autre}
    \begin{itemize}
        \item je ne sais pas: わかりません
        \item chose: もの
        \item journal: しんぶん
        \item lettre: てがみ
        \item intieux (な): ひろかった
        \item pluie: あめ (ton descend)
        \item toujours: いつも
        \item vrai: マル faux: バン
    \end{itemize}
\end{CJK}
\end{document}
